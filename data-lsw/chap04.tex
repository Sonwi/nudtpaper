\chapter{面向异质数据的隐私保护梯度聚合技术}

\section{引言}
联邦学习\cite{mcmahan2017communication}(FL)是由Google在2017年提出的一种新颖的分布式机器学习框架,适用于注重数据隐私的参与方在服务器的调动下,协同完成神经网络的训练。
在工业界,目前已经有很多基于联邦学习的实际应用已经完成落地,Google基于联邦学习为移动端打造的智能输入法预测方案 \cite{hard2018federated},医疗行业基于联邦学习落地的智能诊断和治疗系统 \cite{li2020deepfed},以及微众银行基于联邦学习的构建的风险评估系统 \cite{DBLP:conf/ndss/CaoF0G21}。 
简单来说,FL的核心步骤,是在FL服务提供商的协调下,不同的参与方上传本地训练得到的本地更新(即梯度),然后由服务提供商聚合梯度生成全局模型,最后分发给参与方。在整个过程中,用户的数据始终保持在本地,对比传统的分布式机器学习直接交易数据的方式,提升了数据的隐私性。

然而,一些研究 \cite{geiping2020inverting,zhu2019deep,gao2021privacy} 表明,尽管没有直接上传数据,但是FL仍然存在隐私泄露的风险,服务提供商可以通过用户上传的梯度推断出用户的原始数据集,这违背了一些数据保护法,比如GDPR。
同时也有研究 \cite{zhao2018federated, tuor2021overcoming, yoshida2019hybrid}表明,异质数据(即用户间数据分布不一致)给FL的联合训练准确率带来了较大的挑战。
具体来说,真实场景下的FL,往往会遇到参与方数据分布不一致的情况,这会导致参与方的局部目标函数和全局的目标函数之间出现偏差,从而大大影响经典FL训练得到的全局模型的性能。

%TODO 加一个 权重偏差的图

为了解决FL中的梯度隐私泄露问题,许多基于密码协议的安全聚合方案 \cite{liu2021privacy, aono2017privacy, zhang2020batchcrypt, dong2021flod, hao2021efficient} 被学者们提出来。
比如说,一些方案 \cite{liu2021privacy, aono2017privacy, zhang2020batchcrypt} 使用同态加密(HE)来对用户梯度加密,而服务提供商需要在密文状态下进行梯度的聚合,所以能够很好的保护用户梯度的隐私。
除此之外,一些方案 \cite{hao2021efficient, dong2021flod} 利用安全多方计算技术(MPC)来实现梯度的隐私保护,可以在不泄漏用户梯度的情况下,完成梯度的聚合。
在另一方面,为了解决异质数据带来的挑战,一些方案 \cite{li2020federated, gao2022feddc, ghosh2020efficient, briggs2020federated}对经典的FL平均聚合方法(FedAvg\cite{mcmahan2017communication})做了改良,比如说,方案 \cite{li2020federated} 对用户本地训练过程进行了微调,在本地目标函数中添加了一个正则项,用来限制不同用户梯度之间的偏差。

尽管许多工作都在致力于解决FL中的梯度隐私泄露问题,以及数据异质问题,但是他们往往将两个问题分开讨论。
许多解决梯度隐私泄露问题的工作,都没有考虑到异质数据对FL发起的挑战,在面对异质数据时,性能表现地下。而许多提升异质数据联合训练性能的方案,都没有考虑到梯度的隐私泄露问题,直接使用梯度明文进行聚合。其中文献 \cite{xiong2021privacy} 基于差分隐私(DP)同时考虑了上述两个问题,但是对梯度加入的随机噪声也会影响全局模型的性能。

针对上述研究现状,本章提出了一个兼顾梯度隐私保护与异质数据联合训练准确率的FL框架,该框架(PPFL+HC)以提升异质数据联合训练性能的前沿方案FL+HC \cite{briggs2020federated} 为基础,利用两方安全计算技术(2PC),将FL+HC中涉及到的梯度计算,进行精心的2PC安全协议设计,保证用户本地梯度以及聚合之后的全局梯度,始终对服务提供商保密,以此实现梯度的完全隐私保护。FL+HC在经典的FL流程中,添加了一个层次聚类步骤,将梯度相似的用户划分为一个簇,在一个簇之间联合生成全局模型。
因此PPFL+HC的核心任务即是在密文梯度上,完成高效的、高精度的层次聚类。为了达成这个目标,我们设计了安全高效的梯度间距离的计算算法(包括欧式距离和曼哈顿距离),利用对梯度的随机维度裁剪来提升计算效率,同时通过控制裁剪的比例,保证聚类精度的同时,最大限度的减小计算开销。同时,我们利用伪随机生成技术(PRG \cite{yao1982theory})在不额外提升通信开销的情况下,实现对全局梯度的隐私保护。
同时在真实数据集上进行的实验表明,我们的PPFL+HC可以在保护梯度隐私的前提下,显著提升异质数据的联合训练性能。

本章的组织结构如下:第\ref{4-pre}节介绍了FL中异质数据的类型和影响、方案FL+HC的简单介绍以及基于秘密分享的安全两方计算。
第\ref{4-problem}节介绍了本章的系统模型、威胁模型以及设计目标。
第\ref{4-building}节介绍了一些列基于秘密共享的隐私保护计算模块。
第\ref{4-framework}节提出了支持隐私保护的面向异质数据的FL框架。
第\ref{4-exp}节对方案进行了实验评估。
最后,第\ref{4-conclusion}节对本章进行了总结。

\section{预备知识}\label{4-pre}

\section{问题描述}\label{4-problem}

\section{基于秘密共享的隐私保护计算模块}\label{4-building}

\section{面向异质数据的隐私保护梯度聚合方案}\label{4-framwork}

\section{实验评估}\label{4-exp}

\section{本章小结}\label{4-conclusion}