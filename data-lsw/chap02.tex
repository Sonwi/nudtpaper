\chapter{相关研究}
\label{chap:main}
当前相关研究人员针对联邦学习中的安全隐私威胁,以及异质分布数据带来的挑战等热点问题,已经取得了一系列的研究成果。
本章主要梳理和总结了国内外针对隐私保护联邦学习、拜占庭鲁棒联邦学习以及异质分布数据联邦学习的相关研究成果,
并深入分析了其在隐私性、鲁棒性和执行效率之间的权衡,为本文后续的研究做铺垫。

\section{隐私保护联邦学习}
\label{sec:ppfl}
%TODO 加个全局的思维导图
为了解决联邦学习(Federated Learning,FL)中用户直接上传模型参数造成的数据隐私泄露问题,学者们提出了一系列方案,利用隐私保护技术,在确保正确聚合的前提下,提升联邦学习的安全性。
这些方案使用到的隐私保护技术主要分为四类,分别是:差分隐私(Differential Privacy,DP)\cite{dwork2006differential}、同态加密(Homomorphic Encryption,HE)\cite{gentry2013homomorphic}、安全多方计算(Secure Multi-Party Computation,MPC)\cite{shamir1979share}以及可信执行环境(Trusted Execution Environment,TEE)\cite{sabt2015trusted}。

\subsection{基于DP的隐私保护联邦学习}
差分隐私(DP)通过在数据或者参数中添加噪声的方式,来保护参与方的数据或模型参数隐私,它提供的隐私基本没有带来额外的计算开销,学者们提出了很多基于DP的隐私保护联邦学习方案\cite{agarwal2018cpsgd, choudhury2019differential, dubey2020differentially, geyer2017differentially, hao2019efficient, hao2019towards, hu2020personalized, rodriguez2020federated, triastcyn2019federated, wei2020federated}。举例来说,
Geyer等人\cite{geyer2017differentially}通过向聚合后的全局模型参数添加DP噪声,以此避免恶意参与方推理其它参与方的隐私数据,但是会把用户模型参数直接泄露给聚合服务器。
Hao等人\cite{hao2019efficient}提出的方案与此不同,其将DP噪声添加在用户本地模型参数,以此保证用户参数的隐私。
Shokri等人\cite{shokri2015privacy}首先考虑到深度学习中的数据隐私保护问题,让多个深度学习参与方通过分享一部分模型参数来优化模型表现,并且对分享的参数进行了DP扰动用以保证参数的隐私,但是由于DP噪声的加入,该方案需要在准确率和用户数据隐私性之间做取舍。
为了进一步提升模型的可用性,Abadi等人\cite{abadi2016deep}在使用DP时对隐私成本进行了更加精细的分析,实现了在适度的隐私预算下,以可控的软件复杂度、训练效率和模型成本来训练具有非凸函数的深度神经网络。
然而,Jayaraman等人\cite{jayaraman2019evaluating}在深度学习网络的训练中,对模型准确率和隐私预算进行了细致的量化分析,发现基于DP的隐私保护深度学习方案,基本不能提供有效的准确率和隐私之间的权衡。

\subsection{基于MPC的隐私保护联邦学习}
安全多方计算(MPC)技术可以在保证数据隐私的前提下,协调多个参与方安全计算出函数输出,可以用于联邦学习中用户数据或模型参数的隐私保护。学者们也提出了很多基于MPC的隐私保护联邦学习方案\cite{bonawitz2017practical, mohassel2017secureml, mugunthan2019smpai, reich2019privacy, sharma2019secure, so2020scalable, xu2019hybridalpha, zhu2020privacy}。举例来说,
Bonawitz等人\cite{bonawitz2017practical}首先提出了一种双掩码方案来保护用户上传的模型参数隐私,其中一个掩码用于保护单个用户参数的隐私(相加聚合之后会抵消),另一个掩码用于保证掉线用户的参数隐私。对于任何一个用户参数,只有一个掩码会被去掉,所以可以在不侵犯掉线用户隐私的前提下完成安全聚合,但是这种方案显著增加了用户侧的通信开销。
Mohasse等人\cite{mohassel2017secureml}以及Bell等人\cite{bell2020secure}利用安全多方计算技术(MPC),将数据以秘密共享的方式,外包给双云服务器,通过设计的安全高效的安全两方计算协议(2PC),实现对加密数据的联合训练,但是这种类型的方案需要用户保持一直在线,给整个系统在真实世界中的部署带来了额外的瓶颈。
Xu等人\cite{xu2019hybridalpha}利用秘密共享\cite{shamir1979share}和密钥协商技术\cite{hellman1976new},搭建了支持验证服务器聚合运算的隐私保护联邦学习方案。
对于基于MPC的方案来说,需要克服的困难是进一步优化安全聚合时的计算和通信开销。


\subsection{基于HE的隐私保护联邦学习}
同态加密(HE)可以让服务器在密文上直接计算用户模型参数的聚合结果,保证其解密结果和明文计算一致,这个特性用在联邦学习中可以保护用户上传的模型参数隐私,而又不影响参数的聚合,所以很多基于HE的联邦学习方案\cite{asad2020fedopt, chen2020fedhealth, dong2020eastfly, hao2019efficient, hardy2017private, aono2017privacy, zhang2020batchcrypt, zhang2020privacy, zhao2020smss}被学者们提了出来。举例来说,
Aono等人\cite{aono2017privacy}利用加性同态加密(Additively Homomorphic Encrytion,AHE)加密用户上传的参数,服务器利用HE在密文上实现聚合,通过让私钥在参与方之间公开而对服务器保密的方式,实现对用户参数的隐私保护。
Chen等人\cite{chen2020fedhealth}基于AHE提出了针对可穿戴设备的安全联邦学习框架,移动的可穿戴设备通过HE加密本地模型参数然后上传给服务器完成聚合,最后将聚合密文再分发给可穿戴设备。
Hao等人基于全同态加密(Fully Homomorphic Encryption,FHE)\cite{brakerski2014leveled}提出了隐私保护联邦学习方案。
为了优化同态加密的计算和通信开销,Zhang等人\cite{zhang2020batchcrypt}对用户梯度进行了批量编码,即将多个梯度值编码为一个长整数,再加密为密文上传到聚合服务器。
Zhang等人\cite{zhang2020privacy}使用HE加密本地梯度,同时利用分布式选择随机梯度下降算法减小计算开销。
Ma等人\cite{ma2021privacy}使用HE中CKKS多密钥变种MK-CKKS对参与方持有相同私钥的方式进行了优化,同时修订了多密钥密文线性聚合时的隐私泄露问题,进一步提升了联邦学习的隐私性。


\subsection{基于TEE的隐私保护联邦学习}
可信执行环境(TEE)基于硬件安全的CPU,实现了基于内存隔离的安全计算方式,联邦学习可以将涉及隐私的计算放入TEE,实现对用户数据的隐私保护。学者们提出了一些基于TEE的隐私保护联邦学习方案\cite{tramer2018slalom, mo2019efficient, mo2021ppfl, mo2020darknetz}。举例来说,
Tramer和Boneh\cite{tramer2018slalom}利用因特尔的可信执行环境,将神经网络的中敏感层的训练移入TEE中,利用定制化的硬件实现对用户数据的隐私保护。Mo等人\cite{mo2021ppfl}在此基础上,将所有的神经网络计算移入TEE,并且优化了计算和通信效率。但是基于TEE的方案很难扩展到大规模的应用,因为安全硬件分配给TEE的内存都较小,使得TEE的大规模使用造价昂贵。同时相关研究\cite{van2018foreshadow}表明,基于TEE的隐私保护方案,会受到来自硬件的测信道攻击(Side channel attacks)。

\subsection{基于混合密码技术的隐私保护联邦学习}
由于密码技术有着各自的优点,为了结合各自的长处,一些基于混合密码技术的隐私保护联邦学习方案\cite{choquette2021capc, hao2019efficient, hao2019towards, mugunthan2019smpai, truex2019hybrid, xu2019verifynet, xu2019hybridalpha, zhang2020privacy, zhao2020smss}被学者们提出。举例来说,
文献\cite{xu2019verifynet, zhao2020smss}提出的隐私保护联邦学习,基于同态加密和秘密共享的混合密码技术;
工作\cite{hao2019efficient, hao2019towards}提出的隐私保护联邦学习,基于同态加密和差分隐私的混合密码技术;
研究\cite{mugunthan2019smpai, truex2019hybrid, xu2019hybridalpha}提出的隐私保护联邦学习方案,基于安全多方计算和差分隐私。
%TODO 综述59 了解一下

\subsection{隐私保护联邦学习小结}
以上提出的隐私保护联邦学习方案,仅限于解决联邦学习中出现的隐私威胁,即用户数据或用户模型参数对半诚实服务器或者参与方泄露的威胁。所以其涉及的聚合方案都基于平均线性聚合方案FedAvg\cite{mcmahan2017communication},
正如上文所述,FedAvg算法在如下两种具体场景下,有着较大的局限性。
%而本文认为,隐私安全的联邦学习系统,还须考虑到以下两种典型的场景:
\begin{compactitem}
	\item \textbf{拜占庭节点的干扰:}拜占庭节点指的是联邦学习中的恶意参与方,其可以通过上传精心构造或随机生成的恶意参数,急剧降低联合训练的全局模型准确率,研究\cite{blanchard2017machine}表明,一个拜占庭节点的存在,便可以破坏整个平均线性聚合(FedAvg)的联邦学习方案。在算力和安全性较弱的IoT场景,拜占庭节点将造成较大的安全威胁。
	\item \textbf{异质分布数据的影响:}异质分布数据是指联邦学习中各个参与方之间数据分布不一致的场景,这种场景在医疗金融领域十分常见\cite{li2020federated, gao2022feddc, ghosh2020efficient, briggs2020federated},
	异质分布数据产生的用户梯度之间存在分歧(Divergence),以平均线性聚合FedAvg产生的全局模型,不能同时满足所有参与方对于推理准确率的需求。
%	如果使用FedAvg聚合异质数据产生的模型更新,将会生成准确率较低的全局模型。
\end{compactitem}
以上两种场景,都对线性聚合方案FedAvg提出了挑战,对于拜占庭节点的威胁,联邦学习需要识别恶意参数,并消除其对全局模型参数的影响;对于异质分布数据的场景,联邦学习需要协调目标函数不同的参与方,识别具有相同训练目标的参与方。这都需要更加复杂的聚合方案,也需要更加细致的安全协议设计,去平衡用户模型参数的隐私性和可用性。

\section{拜占庭鲁棒联邦学习}
\label{sec:byzantine}
安全性较弱的联邦学习场景中的拜占庭节点威胁,使得学者们去寻求比FedAvg更加鲁棒的用户模型参数聚合方案\cite{blanchard2017machine, guerraoui2018hidden, yin2018byzantine, DBLP:conf/ndss/CaoF0G21, he2020secure, hashemi2021byzantine, khazbak2020mlguard, liu2021privacy, nguyen2022flame, hao2021efficient, dong2021flod}。在这些方案中,可以根据是否考虑对用户上传参数的隐私保护,划分为直接使用明文模型参数的拜占庭鲁棒聚合方案\cite{blanchard2017machine, guerraoui2018hidden, yin2018byzantine, DBLP:conf/ndss/CaoF0G21},和使用加密或者混淆之后的模型参数的拜占庭鲁棒聚合方案\cite{he2020secure, hashemi2021byzantine, khazbak2020mlguard, liu2021privacy, nguyen2022flame, hao2021efficient, dong2021flod}。考虑用户模型参数的隐私,也就意味着需要进一步在用户模型参数的隐私性和可用性之间做权衡。

%明文
\subsection{基于明文模型参数的鲁棒聚合方案}
%TODO 加一个对与参数 梯度的解释
在McMahan等人\cite{mcmahan2017communication}提出开创性的基于FedAvg的联邦学习方案之后,
Blanchard等人\cite{blanchard2017machine}揭示了拜占庭节点对FedAvg的威胁,并提出了Krum聚合方案来提升联邦学习面对拜占庭节点的鲁棒性,其核心思想是通过计算两两用户之间的欧式距离,选择出距离其它所有用户梯度距离之和最小的用户梯度,作为本轮聚合的全局梯度。
为了让更多的本地梯度为全局梯度做贡献,Guerraoui等人 \cite{guerraoui2018hidden}提出了Bulyan梯度聚合方案,其借鉴了Krum的距离计算方式,然后按照Krum的标准选出了多份梯度,再逐维度取中值作为基准,按照到基准的距离分配权重,聚合选中的梯度产生全局梯度。
这两种计算方案都依赖于所有参与方两两之间的距离计算,对于$N$个参与方来说,其计算复杂度为$O(N^2)$,对于维度较大的梯度来说,计算开销较大。
除此之外,Yin等人\cite{yin2018byzantine}提出了两种逐维度聚合方案,Median 和Trimmed Mean,其聚合方式是对于梯度的第$i$个维度,聚合服务器将所有参与方的这个维度排序,从中取出位于中部的中值(Median)或裁剪掉固定占比的头部尾部数值之后的平均值(Trimmed Mean),作为这个维度的聚合值。
为了抵抗更多前沿的拜占庭攻击,Cao等人\cite{DBLP:conf/ndss/CaoF0G21}提出了一种利用收集小部分根数据(Root dataset)的方式引导信任,在聚合服务器侧维护一个服务端模型,以此过滤拜占庭节点上传的恶意参数。但此方法的鲁棒性,非常依赖于所收集数据的质量。

%密文
\subsection{基于密文模型参数的鲁棒聚合方案}
一些前沿方案同时兼顾到了用户模型参数隐私,和面对拜占庭节点的鲁棒性。
He等人\cite{he2020secure}基于加性秘密共享的安全计算协议,结合Krum的聚合方案的修订方案,实现了在梯度分享值上的拜占庭阶段过滤。
So等人\cite{so2020byzantine}基于Shamir的秘密共享和双掩码机制,改造了Krum聚合方案,实现了对用户模型参数的隐私保护。
为了进一步提升聚合方案的鲁棒性,Khazbak等人\cite{khazbak2020mlguard}利用余弦相似度作为距离度量标准,结合安全多方计算(MPC)技术,同时实现了梯度的隐私保护和安全聚合。
然而以上方案的计算和通信复杂度都随着参与方数量呈平方式增长,主要原因是需要计算每个模型参数,到其它所有参数的距离值。
为了进一步提升密文聚合计算效率,
Liu等人\cite{liu2021privacy}基于同态加密和双云服务器,以梯度的逐维度中值为基准,计算聚合权重对用户梯度进行线性聚合。但是其为了实现一些同态密文上的复杂计算(比如逐维度取中值),其设计的四个安全协议中都将一个用户梯度所有维度,用同一个随机数扰动然后解密,这会泄露用户梯度的分布情况,同时研究\cite{comments}证明了这种分布信息的泄露足以导致所有用户模型参数信息被窃取。
Dong等人\cite{dong2021flod}和Hao等人\cite{hao2021efficient}基于FLTrust\cite{DBLP:conf/ndss/CaoF0G21}根数据集的思想,以此辨别恶意梯度,在2PC上设计了安全有效的计算协议。但是其收集了部分用户的隐私数据,这直接侵犯了用户的数据隐私。
Nguyen等人\cite{nguyen2022flame}针对后门攻击(Backdoor attack)精心设计了鲁棒的聚合方案,并结合了ABY隐私计算框架\cite{demmler2015aby}和隐私保护DBSCAN \cite{bozdemir2021privacy}设计了梯度的隐私保护计算模块,但是其鲁棒聚合方案涉及到2PC上比较耗时的余弦相似度计算和聚类运算,这带来了较大的计算和通信负担。

\subsection{拜占庭鲁棒联邦学习小结}
对于基于明文模型参数的方案来说,其最大的安全威胁来自于半诚实服务器对用户隐私数据的推理\cite{geiping2020inverting},所以需要使用密码技术或扰动技术,对参数进行加密或扰动。
在加密或扰动之后,要面对的问题是隐私性、鲁棒性和计算效率三者之间的权衡,而现有工作都还有一定的缺陷,其本质原因是考虑鲁棒聚合方案时,没有很好的兼顾隐私性带来的计算限制。
本文选择换个出发点,充分考虑隐私性带来的计算限制,设计密文计算友好的聚合方案,在平衡隐私、鲁棒和效率上更进一步。

\section{异质分布数据联邦学习}
\label{sec:noniid}
联邦学习中的异质分布数据,即参与方之间数据分布的非独立同分布(Non-Independent and Identically Distribution,Non-IID)性质,会造成不同用户本地训练产生的参数更新之间发生分歧(Divergence),尤其是对复杂的神经网络训练任务。
当前处理联邦学习中异质分布数据的主流方法分为三类,第一种基于本地数据的调整;第二种基于训练算法的优化;第三种基于系统结构的优化,下面对相关文献进行细节描述。

\subsection{基于数据的异质聚合方案}
由异质分布数据引起的性能下降问题,一种很直观的解决方案就是对参与方的数据分布进行调整,其中数据共享(Data sharing)和数据增强(Data augmentation)\cite{tanner1987calculation}是最主要的两个策略。

首先介绍基于数据共享的方案\cite{zhao2018federated, yoshida2020hybrid, tuor2021overcoming}。
在文献\cite{zhao2018federated}中,聚合服务器持有一份独立同分布(IID)的训练集$\mathcal{D}$,并使用该训练集训练全局模型,同时共享部分全局数据给参与方,与本地数据集一起参与本地训练,以小部分数据集的共享换取全局准确率的提升。
基于数据共享的方案有很明显的缺点,首先收集数据的行为显然违背了隐私保护合规法律,其次对于收集数据的要求是独立同分布,很难应用于实际的联邦学习场景。
然后介绍基于数据增强的方案\cite{duan2019astraea, shin2020xor, yoonfedmix},数据增强\cite{tanner1987calculation}最初是一种通过一些随机变换或知识迁移来增加训练数据多样性的技术,也可用于缓解联邦学习中的本地数据的异质问题。
Duan等人\cite{duan2019astraea}通过收集不同参与方的数据标签分布信息,然后定制化的给出数据增强策略,由此生成的数据和用户本地的数据一起参与训练。
Yoon等人\cite{yoonfedmix}通过对每个参与方收集一部分数据,合并返还给所有参与方,以此减少本地数据的不平衡程度。

综上所述,数据共享和数据增强策略,确实能够缓解参与方之间的数据分布不一致问题,但是这些策略都会直接侵犯用户的数据隐私,存在较大的隐私威胁。

\subsection{基于算法的异质聚合方案}
解决异质分布数据问题的另一种思路是定制化的算法调整,主要的方法有以下三类:本地训练微调(Local fine-tuning)\cite{hanzely2020federated, t2020personalized, huang2021personalized}、本地个性化层(Personalization layer)以及多任务学习(Multi-task learning)
%以及知识蒸馏(Knowledge distillation)。

首先介绍基于本地微调的方案,文献\cite{li2020federated, hanzely2020federated, t2020personalized, huang2021personalized}都对本地的训练过程进行了微调,比如在用户优化目标中添加特定的正则项,用以限制用户模型参数的更新幅度,减小不同参与方之间的本地参数差距,提升联合训练的全局模型性能。此种类型的方案本质上还是追求训练统一的全局模型,在面对比较极端的异质分布数据时,很难让所有的参与方同时保持较高的推理准确率。

基于个性化层的方案,将参与方本地的训练模型划分为共享的基础层和独占的个性化层,在参数聚合时只聚合共享层。
文献\cite{arivazhagan2019federated}提出了典型的个性化联邦学习方案FedPer,其将神经网络中的浅层设置为共享层,用于提取公共的特征,将深层作为独占层便于生成个性化的分类结果,其共享层的参数聚合与FedAvg一致。
与此相反,文献\cite{liang2020think}提出了个性化联邦学习
方案 LG-FEDAVG,其将浅层作为独占层,深层作为共享层,在研究中兼顾了有监督学习、无监督学习以及自监督学习。
基于个性化层的方案,不仅可以提升面对异质分布数据时的推理准确率,还能减小参与方和聚合服务器之间的通信开销。

解决联邦学习异质分布数据问题的另一种方法是将其作为一个多任务学习问题。例如,MOCHA\cite{smith2017federated}是联合多任务学习(FMTL)的代表性框架,首次考虑了联邦学习的通信成本和容错问题。但MOCHA为每个客户端生成了分离但相关的模型,这使得它不适合于非凸优化任务。Corinzia等人\cite{corinzia2019variational}提出了一个使用贝叶斯网络和近似变量推理的FMTL框架VIRTUAL,可以处理非凸模型。他们的方法在几个Non-IID数据集上取得不错的结果,但由于顺序微调,在有大量参与方时很难收敛。

%TODO 知识蒸馏
%知识蒸馏\cite{buciluǎ2006model}是一种从大的教师模型学习小的学生模型的方法,将其应用到FL的主要动机是可以迁移服务器或者其它参与方的模型知识,到特定的用户模型,以此提高异质数据之间的联合训练准确率,

综上所述,基于算法的调整主要集中在用户侧,添加了更多属于用户的个性化操作,以此来面对异质分布数据导致的准确率下降问题。

\subsection{基于聚类的异质聚合方案}
在面对参与方的异质分布数据环境时,基于聚类的方案不局限于生成一份全局模型,它认为一个模型很难学习到所有参与方的数据特征。因此,使用聚类算法将参与方划分到不同的簇,每个簇生成属于本簇的全局模型。
文献\cite{ghosh2022efficient}使用不同参与方的loss值来确定划分的簇,具体来说,聚合服务器维护多个全局模型,在每轮将所有的全局模型分发给参与方,然后参与方利用本地数据获取不同全局模型的loss值,选择loss最低的那个全局模型作为其所属的簇模型,显然这种方式显著增加了通信量和计算量。
Kopparapu等人\cite{kopparapu2020fedfmc}优化了上述算法,仅在特定的轮次进行类别的划分。
除此之外,还有一类基于用户本模型参数相似度来聚类划分的方案\cite{briggs2020federated, sattler2020clustered, dempster1977maximum}。
文献\cite{briggs2020federated}在特定的训练轮次,对所有的参与方上传的梯度进行一次层次聚类,然后后续轮次根据聚类结果,针对每个簇利用FedAvg生成一份全局模型参数。

综上所述,基于聚类的方案不局限于单个全局模型,其对异质分布数据的处理能力比前面提到的单个模型强,且可以和前面的方案结合使用。

\subsection{基于DP的异质聚合方案}
为了同时兼顾模型参数的隐私保护需求以及异质分布数据的处理能力
,有一类基于差分隐私(DP)的方案被学者们提了出来\cite{xiong2021privacy, noble2022differentially}。Xiong等人\cite{xiong2021privacy}提出了2DP-FL方案,它使用双DP的方式为本地模型和全局模型同时添加噪声,以扰动的方式增加异质分布数据的联合训练性能。但是其提供的隐私保证较弱,更适用于算力较弱,需要对隐私保护做出妥协的物联网场景。Noble等人\cite{noble2022differentially}使用DP改造SCAFFOLD,提出了DP-SCAFFOLD方案,致力于用DP保护用户梯度隐私的同时,解决引入噪声带来的模型训练性能下降问题。以上基于DP的方案都需要平衡添加的噪声和梯度数据的可用性,妥协于数据可用性,只能提供一定程度上的隐私保护,无法实现对梯度的强隐私保护。同时此种类型的方案由于噪声的加入,不可避免的影响原始算法的性能。

\subsection{异质分布数据联邦学习小结}
上述提到的针对异质分布数据的联邦学习方案,都在尝试提升异质分布数据联合训练的准确率,但大多忽略了对模型参数的隐私保护问题。
而兼顾模型参数隐私与异质分布数据的方案在隐私性、训练性能以及扩展性上有待提升。
%如果要考虑用户模型参数的隐私性,一些在用户侧的改进可以被第\ref{sec:ppfl}节提到的隐私保护联邦学习所兼容,但是在服务器侧做出的类似聚类的操作,就需要定制安全计算协议来实现模型参数隐私保护。
%因此我们实现了一个兼容Non-IID场景的隐私保护联邦学习训练框架,
%\newpage
\section{本章小结}
本章主要针对隐私保护联邦学习、面向拜占庭容错的联邦学习以及面向异质分布数据的联邦学习相关研究成果展开了综述。
通过梳理代表性的研究成果,本章总结了相关领域所面临的问题和挑战。
总体来说,当前联邦学习中的FedAvg算法面临着拜占庭节点和异质分布数据的挑战,迫切需要更加鲁棒的聚合方案来提升鲁棒性,
同时用户模型参数的隐私问题又限制着这些方案的设计和实现,需要权衡隐私性、安全性和鲁棒性。
本文将围绕拜占庭鲁棒模型参数安全聚合技术,以及异质分布数据模型参数安全聚合技术进行研究。
\settocdepth{subsection} 
