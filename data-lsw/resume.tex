\begin{resume}
\ifreview

\ifismaster
该论文作者在学期间取得的阶段性成果(学术论文等)已满足我校硕士学位评阅相关要求。为避免阶段性成果信息对专家评价学位论文本身造成干扰,特将论文作者的阶段性成果信息隐去。
\else
该论文作者在学期间取得的阶段性成果(学术论文等)已满足我校博士学位评阅相关要求。为避免阶段性成果信息对专家评价学位论文本身造成干扰,特将论文作者的阶段性成果信息隐去。
\fi

\else

\ifisresumebib

	\begin{refsection}[ref/resume.bib]
%		\settoggle{bbx:gbtype}{false}%局部设置不输出文献类型和载体标识符
		\settoggle{bbx:gbannote}{true}%局部设置输出注释信息\\
		\defcounter{maxnames}{6}
		
		\ifisanon{
			\nocite{lsw-nss-anon, lsw-nudt-anon}
		}
		\else{
			\nocite{lsw-nss, lsw-nudt}
		}
		\fi
		
		\printbibliography[heading=subbibliography,title={发表的学术论文}] % 发表的和录用的合在一起
	\end{refsection}

	\begin{refsection}[ref/resume.bib]
	%		\settoggle{bbx:gbtype}{false}%局部设置不输出文献类型和载体标识符
	\settoggle{bbx:gbannote}{true}%局部设置输出注释信息
	\ifisanon{\nocite{lsw-nss-anon, lsw-nudt-anon}}
	\else{\nocite{lsw-nss, lsw-nudt}}
	\fi
	
	\printbibliography[heading=subbibliography,title={在审的学术论文}] % 发表的和录用的合在一起
\end{refsection}
	
	
%	\begin{refsection}[ref/resume.bib]
%		\nocite{ref-8-1-任天令,ref-9-1-Ren}
%		\printbibliography[heading=subbibliography,title={研究成果}]
%	\end{refsection}

\else

  \section*{发表的学术论文} % 发表的和录用的合在一起

\begin{enumerate}[label={[\arabic*]},itemsep=0pt,parsep=0pt,labelindent=26pt,labelwidth=*,leftmargin=0pt,itemindent=*,align=left]
	%[label=\textbf{[\arabic*]},itemindent=*, align=left] %老版本缩进对齐
	
	%\addtolength{\itemsep}{-.36\baselineskip}%缩小条目之间的间距,下面类似
	%\item Xuelun Huang, Shaojing Fu, Yuchuan Luo , Liu Lin. A Novel Location Privacy-Preserving Task Allocation Scheme for Spatial Crowdsourcing[C]. Mobile Multimedia Communications. Springer, Cham, 2021: 304–322.\underline{(EI收录, 检索号: 20214711205654.)}
	
%	\item 第一作者. A Novel Location Privacy-Preserving Task Allocation Scheme for Spatial Crowdsourcing[C]. Mobile Multimedia Communications. Springer, Cham, 2021: 304–322.\underline{(EI收录, 检索号: 20214711205654.)}
	\ifisanon
		\item 第一作者. Privacy-preserving Federated Learning with Hierarchical Clustering to Improve Training on Non-IID Data[C]//Network and System Security: 17th International Conference, NSS 2023, University of Kent, Canterbury, UK, August 14--16, 2023, Proceedings. 2023(EI收录尚未检索)
		
		\item 第一作者. Privacy-preserving Byzantine-robust Federated Learning via Multiparty Homomorphic Encryption[C]//国防科技大学第五届研究生学术活动节. 2022:496-505
	\else
		\item Luo S, Fu S, Luo Y, Liu L, Deng Y and Wang S. Privacy-preserving Federated Learning with Hierarchical Clustering to Improve Training on Non-IID Data[C]//Network and System Security: 17th International Conference, NSS 2023, University of Kent, Canterbury, UK, August 14--16, 2023, Proceedings.(EI收录尚未检索)
		
		\item Luo S, Fu S. Privacy-preserving Byzantine-robust Federated Learning via Multiparty Homomorphic Encryption[C]//国防科技大学第五届研究生学术活动节. 2022:496-505
	\fi
	
	
\end{enumerate}

  \section*{在审的学术论文} % 发表的和录用的合在一起

\begin{enumerate}[label={[\arabic*]},itemsep=0pt,parsep=0pt,labelindent=26pt,labelwidth=*,leftmargin=0pt,itemindent=*,align=left]
	%[label=\textbf{[\arabic*]},itemindent=*, align=left] %老版本缩进对齐
	
	%\addtolength{\itemsep}{-.36\baselineskip}%缩小条目之间的间距,下面类似
	%\item Xuelun Huang, Shaojing Fu, Yuchuan Luo , Liu Lin. A Novel Location Privacy-Preserving Task Allocation Scheme for Spatial Crowdsourcing[C]. Mobile Multimedia Communications. Springer, Cham, 2021: 304–322.\underline{(EI收录, 检索号: 20214711205654.)}
	
	%	\item 第一作者. A Novel Location Privacy-Preserving Task Allocation Scheme for Spatial Crowdsourcing[C]. Mobile Multimedia Communications. Springer, Cham, 2021: 304–322.\underline{(EI收录, 检索号: 20214711205654.)}
	\ifisanon
	\item 第一作者. Privacy-preserving Byzantine-robust Federated Learning against Untargeted Poisoning Attacks. //IEEE Transactions on Services Computing.(CCF A类期刊,一审中)
	
	\else
	\item Luo S, Fu S, Luo Y, Liu L, He D and Qu L. Privacy-preserving Byzantine-robust Federated Learning against Untargeted Poisoning Attacks. //IEEE Transactions on Services Computing.(CCF A类期刊,一审中)
	
	\fi
	
	
\end{enumerate}

\iffalse
\section*{研究成果} % 发表的和录用的合在一起

\begin{enumerate}[label={[\arabic*]},itemsep=0pt,parsep=0pt,labelindent=26pt,labelwidth=*,leftmargin=0pt,itemindent=*,align=left]
	%[label=\textbf{[\arabic*]},itemindent=*, align=left] %老版本缩进对齐
	
	%\addtolength{\itemsep}{-.36\baselineskip}%缩小条目之间的间距,下面类似
	\item 专利:NI211283CN人员流动性的安全预测方法、系统、客户端设备及服务器(申请中)
	\item 专利:NI220234CN 基于契约的众包激励方法、系统、设备和存储介质(申请中)
\end{enumerate}
\fi

\section*{参与项目} % 有就写,没有就删除
\begin{enumerate}[label={[\arabic*]},itemsep=0pt,parsep=0pt,labelindent=26pt,labelwidth=*,leftmargin=0pt,itemindent=*,align=left]
	%[label=\textbf{[\arabic*]},itemindent=*, align=left] %老版本缩进对齐
	%\addtolength{\itemsep}{-.36\baselineskip}%
	\item 国家自然科学基金面上项目:泛在网络环境下的加密数据安全查询技术研究(项目编号:62072466)
	\item 国家自然科学基金青年基金:面向在线诊断的隐私保护及可验证技术研究(项目编号:62102430)
	\item 国家自然科学基金面上项目:智慧城市真值发现中的隐私保护问题研究(项目编号: 61872372)
\end{enumerate}
\fi
\fi
\end{resume}
