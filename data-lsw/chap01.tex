\chapter{绪论}
过去几年联邦学习(Federated Learning,FL)经历了自提出以来的飞速发展,并迅速成为了人工智能领域的研究热点。
作为一种隐私增强的合作学习方案,以梯度间接泄露原始数据的隐私威胁,仍然阻碍着FL的进一步发展。
如何在改进经典FL算法以适用更多应用场景的同时,保证用户上传梯度的隐私和本地数据的安全,成为了当前学术研究的热点问题。
本章首先介绍FL的基本背景,以及解决其中安全隐私威胁的重要意义,然后阐述了本文的主要工作和创新点,最后给出了全文的结构。

\section{研究背景与意义} 
%\texttt{PBFL} and \texttt{PPFL-HC} are our schemes to solve the problems!
联邦学习(FL)的概念在2016年首次被Mcmahan等人\cite{mcmahan2017communication}提出,
其核心理念是让不同的参与方将数据保持在本地,通过交换模型更新(参数或者梯度)的方式协同训练一个全局模型,通过避免直接交易数据的方式,提供了一定的隐私保护。
其注重隐私的分布式训练方式迅速得到了发展,并成为了人工智能领域的研究热点\cite{bhagoji2019analyzing}。其蓬勃的发展主要受益于如下三个事实:(1)机器学习和深度学习技术广泛的应用成功落地;(2)大数据时代爆发式的数据增长;(3)全球数据隐私保护法律法规的制定和实施。

机器学习技术广泛而成功的应用是推动FL发展的主要动力。
在过去的几十年里,机器学习技术在各个领域的一些应用中取得了令人瞩目的成就,如自然语言处理\cite{devlin2018bert}、图像处理\cite{zhu2020neural}和生物识别\cite{yin20193d}。其中最有名的应用是AlphaGo\cite{silver2017mastering}。2016年,AlphaGo以4:1的比分成功击败了9段职业棋手;2017年,它继续以3:0的比分成功击败了当时世界排名第一的围棋选手;而现在,它的继任者,自学成才的AlphaZero,被认为是世界上最好的围棋选手。此外,许多其他应用已经被广泛商业化,包括应用于各种电子产品和门禁系统的人脸识别系统。这些成功的机器学习应用为FL的发展铺平了道路。

大数据的爆发式增长导致越来越多数据孤岛(data islands)的出现,进一步推动了FL的发展。
每天都有大量的数据从社交网络、物联网、智能电网、电子商务、医院、银行系统和其他领域产生\cite{hu2016energy}。这种趋势促进了机器学习的发展,但也给传统的机器学习带来了巨大的挑战,因为大数据通常被不同的组织存储在不同的设备中,形成数据孤岛。
例如,典型的数据孤岛就是不同医院持有的患者医疗数据,单个医院的数据在规模和分布上都有局限性,所以无法训练出高质量的模型。在理想的情况下,所有医院可以自由交易数据,联合所有数据进行模型的训练。但是医疗数据包含非常敏感的个人信息,不允许被随意共享。
在隐私问题的限制下,学习一个全局模型对于传统机器学习来说,变得越来越有挑战性。
而FL作为一种新兴的分布式学习范式,正在让人工智能的研究与应用走出数据孤岛的限制。

数据隐私保护的法律规定推动了FL的快速发展。近年来,许多数据泄露事件大大威胁了用户的数据隐私。例如,2019年,亚马逊云服务上超过5.4亿条Facebook用户的记录被曝光\footnote{https://www.upguard.com/breaches/facebook-user-data-leak.},这引起了严重的社会和法律问题。因此为了保护用户的私人数据,许多法律规定被制定出来,如欧盟的《通用数据保护条例》(General Data Protection Regulation,GDPR)\cite{voigt2017eu},新加坡的《新加坡个人数据保护法》\cite{chik2013singapore},以及美国的《加州隐私权利法》\footnote{https://oag.ca.gov/privacy/ccpa.}。这些法规极大地促进了FL的发展,特别是保护隐私的FL。

然而,相关研究表明FL提供的隐私保证还不够。

此外,FL作为分布式协同训练系统,恶意的参与方造成的影响十分巨大。

同时,FL中的异质数据场景,给FL带来了新的挑战。

\section{联邦学习中安全隐私问题概述}


\section{本文研究内容和创新点}


\section{论文的组织结构}
