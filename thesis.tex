% !TeX document-id = {db2cfe80-8ff6-4784-b205-4bd4c226c9ea}
%!TEX program = xelatex
%!BIB program = biber
%%
%% This is file `thesis.tex',
%% generated with the docstrip utility.
%%
%% The original source files were:
%%
%% nudtpaper.dtx  (with options: `thesis')
%% 
%% This is a generated file.
%% 
%% Copyright (C) 2020 by Liu Benyuan <liubenyuan@gmail.com>
%% 
%% This file may be distributed and/or modified under the
%% conditions of the LaTeX Project Public License, either version 1.3a
%% of this license or (at your option) any later version.
%% The latest version of this license is in:
%% 
%% http://www.latex-project.org/lppl.txt
%% 
%% and version 1.3a or later is part of all distributions of LaTeX
%% version 2004/10/01 or later.
%% 
%% To produce the documentation run the or iginal source files ending with `.dtx'
%% through LaTeX.
%% 
%% Any Suggestions : LiuBenYuan <liubenyuan@gmail.com>
%% Thanks Xue Ruini <xueruini@gmail.com> for the thuthesis class!
%% Thanks sofoot for the original NUDT paper class!
%% 
%1. 规范硕士导言
% \documentclass[master,ttf]{nudtpaper}
%2. 规范博士导言
% \documentclass[doctor,twoside,ttf]{nudtpaper}
%3. 建议使用OTF字体获得较好的页面显示效果
%   OTF字体从网上获得,各个系统名称统一。
%   如果你下载的是最新的(1201)OTF英文字体,建议修改nudtpaper.cls,使用
%   Times New Roman PS Std
% \documentclass[doctor,twoside,otf]{nudtpaper}
%   另外,新版的论文模板提供了方正字体选项FZ,效果也不错哦
% \documentclass[doctor,twoside,fz]{nudtpaper}
%4. 如果想生成盲评,传递anon即可,仍需修改个人成果部分
% \documentclass[master,otf,anon]{nudtpaper}
%
%5. 参考文献若用biblatex生成,则使用biber选项
% \documentclass[master,biber]{nudtpaper}
%
%6. 简历中的论文和成果用biblatex参考文献方式生成,则使用resumebib选项
% \documentclass[master,biber,resumebib]{nudtpaper}
%
%7. 如果是专硕,则使用prof选项
%\documentclass[master,twoside,biber,resumebib,prof,ttf]{nudtpaper}
\documentclass[master,biber,resumebib,prof,ttf]{nudtpaper}
%
% \documentclass[doctor,twoside,biber,resumebib,fz]{nudtpaper}
\addbibresource[location=local]{ref/refs.bib}
\usepackage{mynudt}

\classification{TP957}
\serialno{0123456}
\confidentiality{公开}
\UDC{}
\title{联邦学习中梯度的安全聚合关键技术研究}
\displaytitle{联邦学习中梯度的安全聚合关键技术研究}
\author{罗淞巍}
\zhdate{\zhtoday}
\entitle{Research on Key Techniques for Secure Aggregation of Gradients in Federated Learning}
\enauthor{Songwei Luo}
\endate{\entoday}
\subject{计算机技术}
\ensubject{Computer Technology}
\researchfield{隐私保护联邦学习}
\supervisor{付绍静\quad{}教授}
\cosupervisor{} % 没有就空着
\ensupervisor{Prof. Shaojing Fu}
\encosupervisor{} % 没有就空着
\papertype{工学}
\enpapertype{Engineering}
% 加入makenomenclature命令可用nomencl制作符号列表。

\begin{document}
\graphicspath{{figures/}}
% 制作封面,生成目录,插入摘要,插入符号列表 \\
% 默认符号列表使用denotation.tex,如果要使用nomencl \\
% 需要注释掉denotation,并取消下面两个命令的注释。 \\
% cleardoublepage% \\
% printnomenclature% \\
\maketitle
\frontmatter
\tableofcontents
\listoftables
\listoffigures

%如果不是送审论文,则将true改为false即可
\newif\ifreview\reviewfalse

\midmatter
\begin{cabstract}
联邦学习是一种注重隐私的分布式机器学习范式,它将参与方的数据保留在本地,联合各方的本地模型参数,聚合生成全局模型参数,进而完成联合训练,在解决数据孤岛问题上展现了巨大的潜力。但是相关研究表明,参与方上传的模型参数携带有原始数据的特征,仍然有泄露原始隐私数据的风险。同时,一些具体的场景给经典的模型参数聚合算法(FedAvg)带来了新的挑战,比如众多IoT设备联合训练时可能发生的恶意节点(拜占庭节点)干扰,以及医疗金融等领域联合训练时受到的异质分布数据影响。
这揭示了模型参数的隐私性和其在具体场景下的可用性之间的矛盾。一方面,对模型参数的隐私保护需求需要对参数进行加密或者扰动,以预防潜在的隐私泄露风险;另一方面,一些具体的联邦学习场景需要引入复杂的计算保证模型参数的鲁棒聚合,这进一步提升了对模型参数的可用性需求。如何权衡模型参数的隐私性和日益增长的模型参数可用性需求,是一个非常有挑战性且有实际应用价值的难题。
%经典的联邦学习聚合算法(FedAvg)性能急剧下降。
%同时一些具体的联邦学习场景也给经典的联邦学习聚合算法带来了新的挑战。
%同时更加复杂的应用场景也给联邦学习带来了新的挑战。一方面,对模型参数的隐私保护需求需要对参数进行扰动或者加密,以预防潜在的隐私泄露风险;另一方面,复杂场景下的联邦学习也引入了更加复杂的模型参数聚合算法,如何在保护模型参数隐私的同时,实现高效、鲁棒的联邦学习方案,是联邦学习走向更复杂应用场景必须要解决的难题。

本文的研究内容立足于解决联邦学习中模型参数的隐私泄露问题,分别具体考虑两种场景对模型参数安全高效聚合带来的挑战:$(\rm \romannumeral1)$ 恶意节点(拜占庭节点)的干扰;$(\rm \romannumeral2)$异质分布数据的影响。
论文主要工作与核心贡献如下:

(1)针对联邦学习中的隐私威胁和拜占庭节点威胁,本文提出了一种拜占庭容错的隐私保护联邦学习方案。具体来说,本文首先提出了一种密文计算友好的拜占庭鲁棒聚合算法,然后利用多方同态加密技术设计隐私计算模块,在密文上实现了对拜占庭节点的鲁棒聚合。理论分析结合实验结果表明,方案在密文上完成了对恶意模型参数的过滤,充分利用良性用户的模型参数训练全局模型,实现了较高的推理准确率,同时轻量的拜占庭鲁棒聚合算法也保证了密文计算上的高效性。此外,本文提出的密文计算友好的拜占庭容错聚合算法,理论上支持采用不同的隐私保护技术来实现密文上的鲁棒聚合,具有较好的扩展性。

(2)为了应对联邦学习中异质分布数据带来的挑战,同时兼顾对模型参数的隐私保护,本文提出了一种高效的提升异质分布数据训练准确率的隐私保护联邦学习方案。具体而言,本文基于扩展性较好的引入聚类方法的联邦学习方案,利用秘密共享、伪随机生成以及Diffie-Hellman密钥协商等密码学技术,高效实现了基于聚类方法的安全聚合,同时保证了对模型参数的完全隐私保护。理论分析和实验验证表明,方案在对模型参数提供强隐私保护的同时,大幅提升了异质分布数据联合训练的准确率。此外,本文提出的方案具有较好的扩展性,理论上可以结合其它提升异质分布数据训练性能的方案,赋予其模型参数的隐私保护能力,进一步提升全局模型的推理准确率。

%\begin{compactenum}
%	\item 针对联邦学习中的隐私威胁和拜占庭节点威胁,本文提出了一种拜占庭容错的隐私保护联邦学习方案。具体来说,本文首先提出了一种密文计算友好的拜占庭鲁棒聚合方案,再利用多方同态加密作为隐私计算模块,在密文上实现了对拜占庭节点的鲁棒聚合。理论分析结合实验结果表明,方案实现了对中间模型参数的隐私保护、对拜占庭节点的鲁棒性以及聚合计算的高效性。
%	\item 为了应对联邦学习中异质分布数据带来的挑战,同时兼顾对中间模型参数的隐私保护,本文提出了一种高效的提升异质分布数据训练准确率的隐私保护联邦学习方案。具体而言,本文基于扩展性较好的聚类异质联邦学习方案,利用秘密共享、伪随机生成以及密钥协商等密码技术,高效实现了对中间模型参数的完全隐私保护。理论分析和实验验证表明,方案在对中间模型参数提供强隐私保护的同时,大幅提升了异质分布数据联合训练的准确率。
%\end{compactenum}
\end{cabstract}
\ckeywords{联邦学习;隐私保护;拜占庭鲁棒;异质分布数据;多方同态加密;安全多方计算}

\begin{eabstract}
%Federated learning (FL), a privacy-preserving distributed machine learning paradigm that keeps private data local and performs joint training by exchanging model parameters, showing great potential in solving the problem of data silos. However, some researches show that model parameters still have the risk of leaking private data, while more complex application scenarios also bring new challenges to FL. On the one hand, the privacy protection of model parameters requires perturbation or encryption of parameters to prevent potential privacy leakage risk; on the other hand, FL in complex scenarios also introduces more complex model parameter aggregation schemes, and how to achieve an efficient and robust FL scheme while protecting the privacy of model parameters is necessary for FL to be applied in complex application scenarios.

Federated learning (FL) is a privacy-conscious distributed machine learning paradigm, which keeps the data of the participants locally and aggregates the local model parameters of each party to generate global model parameters for joint training, showing great potential in solving the problem of data silos. However, some researches show that the model parameters uploaded by the participants carry the characteristics of the original data, and there is still a risk of leaking the original private data. Meanwhile, some specific scenarios bring new challenges to the classical model parameter aggregation algorithm (FedAvg), such as the interference of malicious nodes (Byzantine nodes) that may occur when many IoT devices are jointly trained, and the influence of heterogeneous distributed data when jointly trained in fields such as healthcare and finance.
This reveals conflicts between the privacy of model parameters and their usability in specific scenarios. On the one hand, the privacy protection of model parameters requires encryption or perturbation of parameters to prevent potential privacy leakage risks; on the other hand, some specific FL scenarios require complex computations to ensure robust aggregation of model parameters, which further enhances the usability of model parameters. The trade-off between the privacy of model parameters and the increasing model parameter usability is a practical challenge.

This paper is devoted to privacy leakage threat in FL, and specifically considers the challenges posed by two specific scenarios for secure and efficient aggregation of model parameters: $(\rm \romannumeral1)$ interference from malicious participants (Byzantine users); and $(\rm \romannumeral2)$ the impact of heterogeneous distributed data. Specifically, the main work and core contributions of this paper are as follows.

(1) To address the privacy threat and Byzantine threat in FL, this paper proposes a Byzantine-robust privacy-preserving FL scheme. In particular, this paper first proposes a ciphertext computationally friendly Byzantine-robust aggregation scheme, and then implements robust aggregation of Byzantine users over ciphertexts using Multiparty Homomorphic Encryption (MHE) as the privacy building block. The theoretical analysis and experimental results show that the scheme accomplishes the filtering of malicious parameters over ciphertexts, makes full use of the parameters of benign users to train the global model, and thus achieves high inference accuracy. Meanwhile, our lightweight Byzantine-robust aggregation scheme ensures high efficiency over ciphertext computation. In addition, the Byzantine-robust aggregation scheme proposed in this paper is theoretically scalable by using different privacy-preserving techniques to achieve robust aggregation over ciphertexts.

(2) To address the challenges posed by heterogeneous data distribution in FL while protecting the privacy of model parameters, this paper proposes an efficient privacy-preserving FL scheme for improving the training accuracy of heterogeneously distributed data. Specifically, based on the well-expanded clustered heterogeneous FL scheme, this paper efficiently achieves full privacy protection of the model parameters by using cryptographic techniques such as Secret Sharing (SS), pseudo-random generation (PRG), and Diffie-Hellman key agreement protocol. Theoretical analysis and experimental validation show that the scheme substantially improves the accuracy of joint training of heterogeneous distributed data while providing strong privacy protection for the model parameters. In addition, the proposed scheme has good extensibility, which can be combined with other schemes to improve the training performance of heterogeneous data distribution, endow the privacy-preserving ability of its model parameters, and thus further improve the inference accuracy of the global model.

\end{eabstract}
\ekeywords{Federated Learning, Privacy-preserving, Byzantine-robust, Heterogeneous data distribution, Multiparty Homomorphic Encryption, Secure Multi-Party Computation}


\chapter*{符号使用说明}
% 可以根据需要在chapter后加星星/去掉星星

\begin{denotation}

\item[HPC] 高性能计算 (High Performance Computing)
\item[cluster] 集群
\item[Itanium] 安腾
\item[SMP] 对称多处理
\item[API] 应用程序编程接口
\item[PI]	聚酰亚胺
\item[MPI]	聚酰亚胺模型化合物,N-苯基邻苯酰亚胺
\item[PBI]	聚苯并咪唑
\item[MPBI]	聚苯并咪唑模型化合物,N-苯基苯并咪唑
\item[PY]	聚吡咙
\item[PMDA-BDA]	均苯四酸二酐与联苯四胺合成的聚吡咙薄膜
\item[$\Delta G$]  	活化自由能~(Activation Free Energy)
\item [$\chi$] 传输系数~(Transmission Coefficient)
\item[$E$] 能量
\item[$m$] 质量
\item[$c$] 光速
\item[$P$] 概率
\item[$T$] 时间
\item[$v$] 速度

\end{denotation}


%书写正文,可以根据需要增添章节; 正文还包括致谢,参考文献与成果
\mainmatter
\chapter{绪论}

\section{研究背景与意义}
正文内容


\section{联邦学习中梯度的隐私保护问题概述}
正文内容


\section{本文研究内容和创新点}


\section{论文的组织结构}

\chapter{相关研究综述}
\label{chap:main}

\section{隐私保护联邦学习综述}
\label{sec:ppfl}

\section{面向拜占庭容错的联邦学习综述}
\label{sec:byzantine}

\section{面向异质数据的联邦学习综述}
\label{sec:noniid}


\section{本章小结}




%%% Local Variables:
%%% mode: latex
%%% TeX-master: "../main"
%%% End:

\begin{ack}
%总起
	转眼间,研究生阶段即将进入尾声,回忆起三年科大生涯,收获颇丰,感慨良多。在这里我遇到了优秀的老师和友善的同学,邂逅了甜蜜的爱情。对于我的学习与生活,他们给予了无私的帮助与指导,在此表达最衷心的感谢。
%付

	首先,我要衷心感谢我的导师付绍静教授。他是我科研路上的领路人,深耕于学术研究最前线,在学术上给了我具体和深入的指导。他对科研工作热忱而严谨的态度,深深的影响了我。付老师总是工作在第一线,对待学生耐心而细致。每次在研究上遇到了难题,老师总能解开我的疑惑,指引我找到突破的方向。
	除了学习上的指导,付老师还非常关注我们个人的发展。和付老师交流时,他总是不吝分享他的人生经验,拓宽我的视野,给予了我莫大的帮助。
	付老师有着丰富的理论知识、严谨的科研态度以及和善的交流指导方式,在我心里是值得敬爱的良师益友。
	
%罗 柳
	诚挚感谢计算科学系的罗玉川老师和柳林老师。感谢他们在研究工作中对于我细致的指导,每次组会他们都会细致的聆听我的汇报,指出我的不足,指导我逐步完善自己的课题。他们治学严谨,博学多才而又孜孜不倦,是值得我学习的榜样。

%师门
	感谢在师门中相遇的各位,感谢无私分享经验的范书珲师姐、刘开放师兄、张富成师兄、吴文祥师兄、杨璐铭师兄、黄雪伦师姐、冯丹师姐,同级一起奋斗的赵侠男、陈丽杰、王寅秋、杨旭同学,仍在奋斗的黄欣怡、杨钰婷师妹,高鑫文、刘琨鹏师弟等,感谢各位无私的帮助,祝愿各位前程似锦。
%家人

	特别感谢邓晏湘同学,我们因师门结缘,于一起奋斗中相互吸引。有你陪伴的时光里,我过得非常富足,也发生了很多令人惊喜的变化。我们一起觅得良好的学习工作节奏,一起践行健康的饮食作息习惯,一起讨论有趣的书籍近闻消息。我由衷的认为,你点亮了我的生活。祝愿我们携手并进,越来越好。
	
	感谢一直陪伴着我的家人,感谢你们一直以来对我的陪伴、理解与支持。无论遇到什么困难,你们都会鼓励支持我,伴我走过一个又一个人生节点。
	
	感谢同住三年的室友伍林、续永和谢晓达。热心的伍哥给我们都带来了很多便利,开朗的性格也十分感染我。大续也十分关照我。同住三年,非常感谢各位对我的照顾,祝愿各位工作顺利,每天都开开心心。
	
	最后,限于个人的研究水平,在论文撰写过程中难免有不足之处,恳请各位老师与专家批评指正。
%总结

\end{ack}


\cleardoublepage
\phantomsection
\addcontentsline{toc}{chapter}{参考文献}
\ifisbiber
{\hyphenpenalty=500 %
\tolerance=9900 %
\renewcommand{\baselinestretch}{1.35}
\printbibliography[heading=bibliography, title=参考文献]

}
\else
\bibliographystyle{bstutf8}
\bibliography{ref/refs}
\fi

\begin{resume}
\ifreview

\ifismaster
该论文作者在学期间取得的阶段性成果(学术论文等)已满足我校硕士学位评阅相关要求。为避免阶段性成果信息对专家评价学位论文本身造成干扰,特将论文作者的阶段性成果信息隐去。
\else
该论文作者在学期间取得的阶段性成果(学术论文等)已满足我校博士学位评阅相关要求。为避免阶段性成果信息对专家评价学位论文本身造成干扰,特将论文作者的阶段性成果信息隐去。
\fi

\else

\ifisresumebib

	\begin{refsection}[ref/resume.bib]
	\settoggle{bbx:gbtype}{false}%局部设置不输出文献类型和载体标识符
	\settoggle{bbx:gbannote}{true}%局部设置输出注释信息
	\setcounter{gbnamefmtcase}{1}%局部设置作者的格式为familyahead格式
	\nocite{ref-1-1-Yang,ref-2-1-杨轶,ref-3-1-杨轶,ref-4-1-Yang,ref-5-1-Wu,ref-6-1-贾泽,ref-7-1-伍晓明}
	
	\setlength{\biblabelsep}{12pt}
	\printbibliography[env=resumebib,heading=subbibliography,title={发表的学术论文}] % 发表的和录用的合在一起

	\end{refsection}


	\begin{refsection}[ref/resume.bib]
	\settoggle{bbx:gbtype}{false}%局部设置不输出文献类型和载体标识符
	\settoggle{bbx:gbannote}{true}%局部设置输出注释信息
	\setcounter{gbnamefmtcase}{1}%局部设置作者的格式为familyahead格式
	\nocite{ref-8-1-任天令,ref-9-1-Ren}%
	
	\setlength{\biblabelsep}{12pt}
	\printbibliography[env=resumebib,heading=subbibliography,title={研究成果}]

	\end{refsection}

\else

  \section*{发表的学术论文} % 发表的和录用的合在一起

  \begin{enumerate}[label={[\arabic*]}]
  \addtolength{\itemsep}{-.36\baselineskip}%缩小条目之间的间距,下面类似
  \item Yang Y, Ren T L, Zhang L T, et al. Miniature microphone with silicon-
    based ferroelectric thin films. Integrated Ferroelectrics, 2003,
    52:229-235. (SCI 收录, 检索号:758FZ.)
  \item 杨轶, 张宁欣, 任天令, 等. 硅基铁电微声学器件中薄膜残余应力的研究. 中国机
    械工程, 2005, 16(14):1289-1291. (EI 收录, 检索号:0534931 2907.)
  \item 杨轶, 张宁欣, 任天令, 等. 集成铁电器件中的关键工艺研究. 仪器仪表学报,
    2003, 24(S4):192-193. (EI 源刊.)
  \item Yang Y, Ren T L, Zhu Y P, et al. PMUTs for handwriting recognition. In
    press. (已被 Integrated Ferroelectrics 录用. SCI 源刊.)
  \item Wu X M, Yang Y, Cai J, et al. Measurements of ferroelectric MEMS
    microphones. Integrated Ferroelectrics, 2005, 69:417-429. (SCI 收录, 检索号
    :896KM.)
  \item 贾泽, 杨轶, 陈兢, 等. 用于压电和电容微麦克风的体硅腐蚀相关研究. 压电与声
    光, 2006, 28(1):117-119. (EI 收录, 检索号:06129773469.)
  \item 伍晓明, 杨轶, 张宁欣, 等. 基于MEMS技术的集成铁电硅微麦克风. 中国集成电路,
    2003, 53:59-61.
  \end{enumerate}

  \section*{研究成果} % 有就写,没有就删除
  \begin{enumerate}[label=\textbf{[\arabic*]}]
  \addtolength{\itemsep}{-.36\baselineskip}%
  \item 任天令, 杨轶, 朱一平, 等. 硅基铁电微声学传感器畴极化区域控制和电极连接的
    方法: 中国, CN1602118A. (中国专利公开号.)
  \item Ren T L, Yang Y, Zhu Y P, et al. Piezoelectric micro acoustic sensor
    based on ferroelectric materials: USA, No.11/215, 102. (美国发明专利申请号.)
  \end{enumerate}
\fi
\fi
\end{resume}

% 最后,需要的话还要生成附录,全文随之结束。
\appendix
\backmatter
% TeX
\chapter{模板提供的希腊字母命令列表}

大写希腊字母:
\begin{table}[htbp]
\centering
\begin{tabular}{llll}
\toprule
$\Gamma$~\verb|\Gamma| & $\Lambda$~\verb|\Lambda| & $\Sigma$~\verb|\Sigma| & $\Psi$~\verb|\Psi| \\
$\Delta$~\verb|\Delta| & $\Xi$~\verb|\Xi| & $\Upsilon$~\verb|\Upsilon| & $\Omega$~\verb|\Omega| \\
$\Theta$~\verb|\Theta| & $\Pi$~\verb|\Pi| & $\Phi$~\verb|\Phi| & \\
\midrule
$\varGamma$~\verb|\varGamma| & $\varLambda$~\verb|\varLambda| & $\varSigma$~\verb|\varSigma| & $\varPsi$~\verb|\varPsi| \\
$\varDelta$~\verb|\varDelta| & $\varXi$~\verb|\varXi| & $\varUpsilon$~\verb|\varUpsilon| & $\varOmega$~\verb|\varOmega| \\
$\varTheta$~\verb|\varTheta| & $\varPi$~\verb|\varPi| & $\varPhi$~\verb|\varPhi| & \\
\bottomrule
\end{tabular}
\end{table}

小写希腊字母:
\begin{table}[htbp]
\centering
\begin{tabular}{llll}
\toprule
$\alpha$~\verb|\alpha| & $\theta$~\verb|\theta| & $o$~\verb|o| & $\tau$~\verb|\tau| \\ 
$\beta$~\verb|\beta| & $\vartheta$~\verb|\vartheta| & $\pi$~\verb|\pi| & $\upsilon$~\verb|\upsilon| \\ 
$\gamma$~\verb|\gamma| & $\iota$~\verb|\iota| & $\varpi$~\verb|\varpi| & $\phi$~\verb|\phi| \\ 
$\delta$~\verb|\delta| & $\kappa$~\verb|\kappa| & $\rho$~\verb|\rho| & $\varphi$~\verb|\varphi| \\ 
$\epsilon$~\verb|\epsilon| & $\lambda$~\verb|\lambda| & $\varrho$~\verb|\varrho| & $\chi$~\verb|\chi| \\ 
$\varepsilon$~\verb|\varepsilon| & $\mu$~\verb|\mu| & $\sigma$~\verb|\sigma| & $\psi$~\verb|\psi| \\ 
$\zeta$~\verb|\zeta| & $\nu$~\verb|\nu| & $\varsigma$~\verb|\varsigma| & $\omega$~\verb|\omega| \\ 
$\eta$~\verb|\eta| & $\xi$~\verb|\xi| & $\varkappa$~\verb|\varkappa| & $\digamma$~\verb|\digamma| \\ 
\midrule
$\upalpha$~\verb|\upalpha| & $\uptheta$~\verb|\uptheta| & $\mathrm{o}$~\verb|\mathrm{o}| & $\uptau$~\verb|\uptau| \\ 
$\upbeta$~\verb|\upbeta| & $\upvartheta$~\verb|\upvartheta| & $\uppi$~\verb|\uppi| & $\upupsilon$~\verb|\upupsilon| \\ 
$\upgamma$~\verb|\upgamma| & $\upiota$~\verb|\upiota| & $\upvarpi$~\verb|\upvarpi| & $\upphi$~\verb|\upphi| \\ 
$\updelta$~\verb|\updelta| & $\upkappa$~\verb|\upkappa| & $\uprho$~\verb|\uprho| & $\upvarphi$~\verb|\upvarphi| \\ 
$\upepsilon$~\verb|\upepsilon| & $\uplambda$~\verb|\uplambda| & $\upvarrho$~\verb|\upvarrho| & $\upchi$~\verb|\upchi| \\ 
$\upvarepsilon$~\verb|\upvarepsilon| & $\upmu$~\verb|\upmu| & $\upsigma$~\verb|\upsigma| & $\uppsi$~\verb|\uppsi| \\ 
$\upzeta$~\verb|\upzeta| & $\upnu$~\verb|\upnu| & $\upvarsigma$~\verb|\upvarsigma| & $\upomega$~\verb|\upomega| \\ 
$\upeta$~\verb|\upeta| & $\upxi$~\verb|\upxi| & & \\ 
\bottomrule
\end{tabular}
\end{table}

希腊字母属于数学符号类别,请用\verb|\bm|命令加粗,其余向量、矩阵可用\verb|\mathbf|。


\end{document}
